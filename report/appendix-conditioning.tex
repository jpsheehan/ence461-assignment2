\subsection{Conditioning}
\label{ap:conditioning}

The input impedance and gain for the non-inverting amplifier circuit are given by equations \ref{eq:zi} and \ref{eq:gain} respectively.
\begin{align}
	Z_i &= \frac{R_1 (R_2 + R_3)}{R_1 + R_2 + R_3} \label{eq:zi} \\
	A_v &= 1 + \frac{R_1}{R_2 + R_3}\label{eq:gain}
\end{align}

The source impedance is specified as being less than $\SI{100}{\ohm}$.
A input impedance that is at least an order of magnitude higher than the source impedance is required for maximum voltage transfer and minimum current drain.
$Z_i = \SI{1}{\kilo\ohm}$ was decided upon, because it is 10 times higher than the worst-case source impedance.
Equations \ref{eq:r1} and \ref{eq:r2} show the derived formulae for the resistors based upon the given $Z_i$ and $A$ values.
\begin{align}
	R_1 &= Z_i A \label{eq:r1} \\
	R_2 + R_3 &= \frac{Z_i A}{A - 1} \label{eq:r2} \\[1em]
	\therefore R_1 = \SI{130}{\kilo\ohm}, R_2 &= \SI{920}{\ohm}, R_3 = \SI{8.2}{\ohm} \nonumber
\end{align}

Thermal calculations (\textcolor{red}{add reference to appendix}) show that the op-amps will supply minimal current into the microprocessor's ADC inputs.
Thus their power dissapation is relatively low.
The power dissapation is a function of the difference between supply voltage to output voltage, output voltage difference to ground (for single supply devices) and the output current.
With low power dissapation, there is a comfortable margin between the ambient temperature of $\SI{105}{\degreeCelsius}$, the junction operating temperature of $\SI{125}{\degreeCelsius}$, and the device absolute maximum of $\SI{150}{\degreeCelsius}$.
Additionally the worst case quiescent current is $\SI{150}{\micro\ampere}$.
