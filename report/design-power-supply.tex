\subsection{Power Supply}
Current consumption analysis determined 60 - 90 mA and 77-100 mA are required for the ADC and digital sections respectively. (Depending on whether indvidual control of ADC blocks is required, and maximum versues typical buss transceiver consumption.) 
The MIC5202-5.0YMM was selected to provide an independant 5V 100mA output, with disable function, to each section; thus providing the smallest physical footprint and zero ADC current consumption in low power mode.

Opamps are 5V single supply rail devices , a 2.5V (half rail) reference is required and we will assume, at this stage, microcontroller will be a 3.3V device. 
Therefore multiple supply rails are needed and two options are considered: Option 1, two seperate regulators; Option 2, a signle regulratot wiht 5V and 3V3 outputs. 
Additionally the the 5V and/or half rail may need to be conencted to the micro ADC references, but is is assumed that the microprocessr current draw is insiginificant.

\subsection{OpAmp Supply}
A voltage divider off the 5V ADC rail, with voltage follower op amp, provides the 2.5V half rail reference supply. Divider current is 0.5mA, balencing the conflicting requiremtns of low quiencent current with low value resistors to reduce thermal noise (associated with high ambient temperatures) in the ADC section

\subsection{MCU Supply}


Requirements: Linear device,
5V output
>= 105 degrees
6V input.
RoHS compliant
Surface mounted device.
Low dropout voltage at full load. 
Minimum additional components.
Missing parameter: ouput current.

Output current calculations
12 ADC channels and max current 2.7mA. Round up to 3mA gives some overhead.
Further opamp for the 2.5 volt reference gives: 13 x 3mA = 39mA.

Using the Digikey part selector 
RoHS datasheet, active part, T operating 125 degrees, Vin max = 10V (allow some overhead for 6V input) gives:
1)
ADP1720ARMZ-R7
Datasheet ADP1720ARMZ-R7
Which has ouput current 50mA, VDO 275mV, -40 to 125 degree C
2)
Digikey 296-1364-2-ND, IC REG LINEAR 5V 100MA 8SOIC,
%\url{http://www.ti.com/general/docs/suppproductinfo.tsp?distId=10\&gotoUrl=http\%3A\%2F\%2Fwww.ti.com\%2Flit\%2Fgpn\%2Fua78l}
3)
Digikey TC1054-5.0VCT713CT-ND, TC1054-5.0VCT713 
%Datasheet \url{http://ww1.microchip.com/downloads/en/DeviceDoc/21350E.pdf}

% TC1054-5.0VCT713‎ was selected because
It is rated to 125 degrees, and ambiant is 105 degrees
Low drop out voltage (supply is  6V, 5V is required leaving minimal margin) typically 85mV max 120mV
It provies control pins. These may be required to shutdown the ADC amps into low power, and provide a 5V supply status.
Note that the ramp up time from shutdown to active is less than 100uSec.
There are 100 and 150mA versions so this part can be uprated if larger current is required in the future.

Thermal consderations.
PD worst case operating = (Vin - Vout) x Imax => (6.5-5)*50 = 75mW
PD max = (TJ - TA)/ thetaJA => (125-105) / 220 = 90.90mW
Thus PDMax > PD worst therfore device is suitable from thermal perspective.
However vin max = PD max/Imax + Vo => 90.909mW/50mA + 5 = 6.81V max.
Thus the supply input voltage must not exceed 6.8V, however Vin max is 6.5V.

93 units are required (assuming there is a need for 5V in the master unit)
Purhase of 100 units is reduces the cost by 20C and gives seven spares
100x 0.46340 = \$46.34
Using the speciifcation of $\SI{2.7}{\milli\ampere}$ per omp, rounding up to $\SI{3}{\milli\ampere}$ to provide safey margin, across thirteen devices gives total consumption of $\SI{39}{\milli\ampere}$ 
A $\SI{50}{\milli\ampere}$ voltage regliuator

\subsection{Microprocessor Supply}
Requirements
Linear devie since ADC is it main function.
3.3V output
>= 105 degrees
6V input.
RoHS compliantt
SMD
Minimum additional components
Mising parameter: ouput current
Output current calulations

Assume at this stage the micro and 3V3 periphials will comsume 150mA in total (micro, LED x 2, possbible buss chips)

Using the Digikey part sselector 
RoHS datasheet, active part, 125 degrees, Vin max = 10V (allows overhead for 6V input) gives 

Selected Digikey 296-11021-1-ND, TPS76333DBVR
% http://www.ti.com/lit/ds/symlink/tps763.pdf?HQS=TI-null-null-digikeymode-df-pf-null-wwe&ts=1590036136348
It is known that there are 6V versions and also 250mA versions whcih may have a similar form factor, so could be changed if higher Iout is required.

Thermal consderations.
PD worst case operating = (Vin - Vout) x Imax => (6.5-3.3)V * 150mA = 480mW
PD max = (TJ - TA)/ thetaJA => (125-105) / 205.3 = 97.41 mW
Thus PDMax < PD worst therfore device is not suitable.

Checking several devices (TPS76633, TPD706-150 and others) analysis shows the Digikey 296-19675-1-ND, TPS71733DCKR is suitable.

Tj = Tt + gamma?jt * PD max => 105 + 2.7*480mW = 106.296 less than Toperating max = 125 
Tj = Tb + gamma?jb * PD max => 105 + 40.9* 480mW = 124.632 degress just less than Toperating max = 125 
This is on the limit at 6.5V input 
At 6.0 Vin Tj = 121 degrees - still less than Toperating max = 125 
VDO = 300mV worst case.
Price break for 100 uniots thus 100 x 1.00190 =  \$100.19 