\subsection{Microcontroller}
\label{sec:microcontroller}

A 32-bit microcontroller (MCU) with at least 12 12-bit ADCs was required to communicate the current readings to the masterboard.
This MCU must communicate with masterboard to send readings every $\SI{100}{\milli\second}$.
Each slave must be time-synced with all other slaves with a deviation of no more than $\SI{1}{\micro\second}$.
The MCU must also be able to withstand temperatures of up to $\SI{105}{\degreeCelsius}$ and have support for the CAN protocol.

To meet these requirements the MKE16Z32VLD4 was chosen.
It has an ARM Cortex-M0+ core, a $\SI{48}{\mega\hertz}$ internal clock, $\SI{32}{\kilo\byte}$ of FLASH memory, and 12 SAR ADCs.
It is also able to run in environments of up to $\SI{105}{\degreeCelsius}$ and has a small 44-pin LQFP footprint.

Its hardware CAN support also requires an external CAN receiver.
This is discussed in section \ref{sec:communications}.

The MCU is programmed via a standard IDC connector ($\SI{1.27}{\milli\metre}$ pitch) via the SWD protocol.
This is how the device will be initially programmed and debugged.
When it is first programmed, a unique address will be stored in its memory. This is used for the CAN protocol.
It is quite feasible to expect that the device will be able to be programmed via the CAN bus.
This allows software updates to be uploaded to all slaves at the same time quickly.

An LED is provided to give some visual feedback and aid in debugging.
It is driven by a P-MOSFET and is powered by the $\SI{6}{\volt}$ rail so it does not draw current from the local regulator.

For the master-board the same microcontroller could be used.
This is because it is high-spec anyway due to the CAN support.
