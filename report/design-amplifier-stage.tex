\subsection{Signal Conditioning}

The process of converting the voltage seen at the current transducer to a digital reading involves three stages (figure \ref{fig:conditioning}).
The gain stage amplifies the voltage of the signal up to a reasonable level.
The filtering stage removes any high frequency noise.
Finally, the sampling stage performs the analog to digital (ADC) conversion.
\begin{figure}[H]
	\centering
	\includegraphics[width=0.8\textwidth]{cat}
	\caption{The signal conditioning process.}
	\label{fig:conditioning}
\end{figure}

\subsubsection{Gain Stage}

The current transducers output a $\pm \SI{25}{\milli\volt}$ signal centered around $\SI{2.5}{\volt}$.
This signal must be amplified to $\pm \SI{2.5}{\volt}$ using an operational amplifier with a gain of 100.
However, the filtering stage attenuates the output voltage by a factor of $\sqrt{2}$.
This mu
Therefore, the gain of the amplifier stage must be 141 for the voltage at the ADC input to have a full $\SI{5}{\volt}$ range.

To amplify the signal from the current transducer, a non-inverting op-amp circuit (figure \ref{fig:non-inverting-op-amp}) was used.
12 of these non-inverting amplifier circuits are required for each daughterboard.
The LMV324 was selected as it has four op-amp circuits in one package.
Resistor values of $R_1 = \SI{130}{\kilo\ohm}, R_2 = \SI{920}{\ohm}, R_3 = \SI{8.2}{\ohm}$ were chosen to give a gain of $141$ and an input impedance of $\SI{1}{\kilo\ohm}$ for each amplifier circuit.

The half-rail voltage of $\SI{2.5}{\volt}$ is provided by a virtual ground op-amp circuit (figure \ref{fig:half-supply}).
An LMV321 was used for this circuit.
High value resistors are required in the divider (typically $\SI{100}{\kilo\ohm}$ to prevent excessive static current drain.
However, these produce greater thermal noise (especially at high ambient temperatures) which is undesirable for ADC applications.
Therefore, two $\SI{2.5}{\kilo\ohm}$ resistors balance the noise and have a static current drain of $\SI{1}{\milli\ampere}$.

\begin{figure}[ht]
\centering

\begin{subfigure}[c]{0.45\textwidth}
	\centering
	\includegraphics[width=\textwidth]{gain_circuit}
	\caption{The non-inverting amplifier circuit.}
	\label{fig:non-inverting-op-amp}
\end{subfigure}
\hfill
\begin{subfigure}[c]{0.45\textwidth}
	\centering
	\vfill
	\includegraphics[width=\textwidth]{virtual_ground_circuit}
	\vfill
	\caption{The virtual ground circuit.}
	\label{fig:half-supply}
\end{subfigure}

\caption{The operational amplifer circuits used.}
\end{figure}

The op-amps were required to switch to a low power mode when no ADC readings are taking place.
The project requirements favour greater control over which op-amps can be turned on and off.
Due to our op-amp selection, they must be turned off in blocks of 4.
Therefore, it was decided that a CMOS inverter IC be used to drive the positive supply rail to ground.
The main drawback of this method is the start-up time to enable the op-amps.
The sum of the time taken to switch the op-amp back on is $\SI{50}{\micro\second}$ (this is a phase angle of $\SI{7.2}{\degree}$ with respect to the AC signal).

\subsubsection{Filtering Stage}

A low-pass filter circuit was designed to attenuate frequencies above the expected $\SI{400}{\hertz}$.
This was done with a simple RC filter (figure \ref{fig:filter}) as the cut-off frequency is less than $\SI{100}{\kilo\hertz}$.
The filter was designed to have a cut-off frequency of $\SI{600}{\hertz}$ so that frequencies above $\SI{400}{\hertz}$ would be attenuated.
The component values were $C = \SI{0.47}{\micro\farad}, R = \SI{560}{\ohm}$.

A drawback of this filter is that the output voltage is $V_{in} \sqrt{2}$.
In this case, the maximum voltage the filter can output is $\SI{3.54}{\volt}$.
To acheive maximum output of $\SI{5}{\volt}$, the gain of the amplifier stage would need to be $141$ and the voltage would need to have a $\pm\SI{3.54}{\volt}$ range, centered around $\SI{2.5}{\volt}$.
Due to the requirements, this is not possible as the output voltage of the amplifier saturates at the supply rail voltage.

\subsubsection{Sampling Stage}

